\nonstopmode{}
\documentclass[a4paper]{book}
\usepackage[times,inconsolata,hyper]{Rd}
\usepackage{makeidx}
\usepackage[utf8]{inputenc} % @SET ENCODING@
% \usepackage{graphicx} % @USE GRAPHICX@
\makeindex{}
\begin{document}
\chapter*{}
\begin{center}
{\textbf{\huge Package `HMLET'}}
\par\bigskip{\large \today}
\end{center}
\inputencoding{utf8}
\ifthenelse{\boolean{Rd@use@hyper}}{\hypersetup{pdftitle = {HMLET: What the Package Does (Title Case)}}}{}
\begin{description}
\raggedright{}
\item[Type]\AsIs{Package}
\item[Title]\AsIs{What the Package Does (Title Case)}
\item[Version]\AsIs{0.1.0}
\item[Author]\AsIs{Alireza Kazemi}
\item[Maintainer]\AsIs{Alireza Kazemi }\email{alireza.kzmi@gmail.com}\AsIs{}
\item[Description]\AsIs{MLET uses machine learning algorithm to analyze Eye-Tracking
data.}
\item[License]\AsIs{MIT + file LICENSE}
\item[Imports]\AsIs{combinat, dplyr, miceadds, purrr, rray, tidyr}
\item[Author@R]\AsIs{person(``Alireza'', ``Kazemi'', email = ``Alireza.kzmi@gmail.com'',
role = c(``aut'', ``cre''))}
\item[Encoding]\AsIs{UTF-8}
\item[RoxygenNote]\AsIs{7.2.1}
\item[NeedsCompilation]\AsIs{no}
\end{description}
\Rdcontents{\R{} topics documented:}
\inputencoding{utf8}
\HeaderA{ClusterStats\_HMLET}{Generate unique random permutations}{ClusterStats.Rul.HMLET}
%
\begin{Description}\relax
Generate unique random permutations
\end{Description}
%
\begin{Usage}
\begin{verbatim}
ClusterStats_HMLET(data, paired = T, detailed = F, threshold_t = NA)
\end{verbatim}
\end{Usage}
\inputencoding{utf8}
\HeaderA{ComputeSubjectLevelPerm\_HMLET}{Generate unique random permutations}{ComputeSubjectLevelPerm.Rul.HMLET}
%
\begin{Description}\relax
Generate unique random permutations
\end{Description}
%
\begin{Usage}
\begin{verbatim}
ComputeSubjectLevelPerm_HMLET(labels, n = 1)
\end{verbatim}
\end{Usage}
\inputencoding{utf8}
\HeaderA{ComputeTValues\_HMLET}{Generate unique random permutations}{ComputeTValues.Rul.HMLET}
%
\begin{Description}\relax
Generate unique random permutations
\end{Description}
%
\begin{Usage}
\begin{verbatim}
ComputeTValues_HMLET(respTime, paired = TRUE)
\end{verbatim}
\end{Usage}
\inputencoding{utf8}
\HeaderA{CreateTimeBinData\_HMLET}{Collapse time points in specified time bins}{CreateTimeBinData.Rul.HMLET}
%
\begin{Description}\relax
Collapse time points in specified time bins
\end{Description}
%
\begin{Usage}
\begin{verbatim}
CreateTimeBinData_HMLET(
  data,
  groupingColumns = NULL,
  timeBinWidth = 250,
  timeMax = 3000,
  FixatedOn,
  timepoint = "timepoint",
  AOIs = NULL,
  timeForward = T,
  aggregateFun = mean
)
\end{verbatim}
\end{Usage}
\inputencoding{utf8}
\HeaderA{ExportDataForMATLAB\_HMLET}{Export to CSV file for MATLAB GUI}{ExportDataForMATLAB.Rul.HMLET}
%
\begin{Description}\relax
Export to CSV file for MATLAB GUI
\end{Description}
%
\begin{Usage}
\begin{verbatim}
ExportDataForMATLAB_HMLET(
  data,
  ID = "ID",
  trial = "trial",
  timepoint = "timepoint",
  timeMax = 3000,
  samplingDuration,
  timeForward = T,
  fixation,
  condition,
  testName = NULL,
  gazeX,
  gazeY,
  gazeXRelative = NULL,
  gazeYRelative = NULL,
  miscVars = NULL,
  fileName = "ETDataforMATLAB.csv",
  path = getwd()
)
\end{verbatim}
\end{Usage}
\inputencoding{utf8}
\HeaderA{FilterClustersTimepoints\_HMELT}{Generate unique random permutations}{FilterClustersTimepoints.Rul.HMELT}
%
\begin{Description}\relax
Generate unique random permutations
\end{Description}
%
\begin{Usage}
\begin{verbatim}
FilterClustersTimepoints_HMELT(data, clusterInf = clusterInf)
\end{verbatim}
\end{Usage}
\inputencoding{utf8}
\HeaderA{FindClusters\_HMLET}{Generate unique random permutations}{FindClusters.Rul.HMLET}
%
\begin{Description}\relax
Generate unique random permutations
\end{Description}
%
\begin{Usage}
\begin{verbatim}
FindClusters_HMLET(tValues, threshold_t = threshold_t)
\end{verbatim}
\end{Usage}
\inputencoding{utf8}
\HeaderA{GenerateRandomData\_HMLET}{Generate unique random permutations}{GenerateRandomData.Rul.HMLET}
%
\begin{Description}\relax
Generate unique random permutations
\end{Description}
%
\begin{Usage}
\begin{verbatim}
GenerateRandomData_HMLET(
  tMax = 20,
  effectOffset = 5,
  trialNum = 40,
  subjNum = 20,
  effectSize = 0.2
)
\end{verbatim}
\end{Usage}
\inputencoding{utf8}
\HeaderA{PermutationTestDataPrep\_HMLET}{Return a data frame compatible for permutation tests routine}{PermutationTestDataPrep.Rul.HMLET}
%
\begin{Description}\relax
Return a data frame compatible for permutation tests routine
\end{Description}
%
\begin{Usage}
\begin{verbatim}
PermutationTestDataPrep_HMLET(
  data,
  ID = "ID",
  trial,
  timepoint,
  condition,
  gazeInAOI,
  conditionLevels = NULL,
  targetAOI = NULL,
  testName = NULL
)
\end{verbatim}
\end{Usage}
%
\begin{Arguments}
\begin{ldescription}
\item[\code{testName}] Name of this data -- can be used as condition name or test names to compare permutation test results between different tests/conditions later
\end{ldescription}
\end{Arguments}
\inputencoding{utf8}
\HeaderA{PermutationTest\_HMLET}{Permutation Tests General Routine}{PermutationTest.Rul.HMLET}
%
\begin{Description}\relax
Permutation Tests General Routine
\end{Description}
%
\begin{Usage}
\begin{verbatim}
PermutationTest_HMLET(
  data,
  samples = 2000,
  paired = T,
  permuteTrialsWithinSubject = F,
  threshold_t = NA
)
\end{verbatim}
\end{Usage}
\inputencoding{utf8}
\HeaderA{PlotNullDistribution\_HMLET}{Plot Null Distribution}{PlotNullDistribution.Rul.HMLET}
%
\begin{Description}\relax
Plot Null Distribution
\end{Description}
%
\begin{Usage}
\begin{verbatim}
PlotNullDistribution_HMLET(resultList)
\end{verbatim}
\end{Usage}
\inputencoding{utf8}
\HeaderA{PlotTemporalGazeTrends\_HMLET}{Plot temporal gaze trends}{PlotTemporalGazeTrends.Rul.HMLET}
%
\begin{Description}\relax
Plot temporal gaze trends
\end{Description}
%
\begin{Usage}
\begin{verbatim}
PlotTemporalGazeTrends_HMLET(
  resultList,
  showDataPointNumbers = T,
  gazePropRibbonAlpha = 0.1,
  clusterFillColor = "#CC9933",
  clusterFillAlpha = 0.5,
  pointSize = 1,
  pointAlpha = 0.7,
  pointFatten = 3,
  testNameOrder = NULL,
  conditionOrder = NULL,
  onlySignificantClusters = T,
  clusterData = NULL
)
\end{verbatim}
\end{Usage}
%
\begin{Arguments}
\begin{ldescription}
\item[\code{resultList}] can be a dataframe of the data that is already prepared by PrepareMLETData\_HMLET or a list that is the result of PermutationTest\_MLET
\end{ldescription}
\end{Arguments}
\inputencoding{utf8}
\HeaderA{PreprocessRawData\_HMLET}{Pre-Process data to extract AOIs}{PreprocessRawData.Rul.HMLET}
%
\begin{Description}\relax
Should be Completed later
\end{Description}
%
\begin{Usage}
\begin{verbatim}
PreprocessRawData_HMLET(
  data,
  ID = "ID",
  trial = "trial",
  timepoint = "timepoint",
  GazeX = "GazeX_Relative",
  GazeY = "GazeY_Relative",
  AOINames = NULL,
  fileName = "ETDataforMATLAB.csv",
  path = getwd()
)
\end{verbatim}
\end{Usage}
\inputencoding{utf8}
\HeaderA{RemoveIncompleteTimePoints\_HMLET}{Remove Time points with missing conditions}{RemoveIncompleteTimePoints.Rul.HMLET}
%
\begin{Description}\relax
This function is currently only applicable for within participant manipulations
\end{Description}
%
\begin{Usage}
\begin{verbatim}
RemoveIncompleteTimePoints_HMLET(data)
\end{verbatim}
\end{Usage}
\inputencoding{utf8}
\HeaderA{SubjectLevelPermutationTestWithin\_HMLET}{SubjectLevelPermutationTestWithin\_HMLET}{SubjectLevelPermutationTestWithin.Rul.HMLET}
%
\begin{Description}\relax
SubjectLevelPermutationTestWithin\_HMLET
\end{Description}
%
\begin{Usage}
\begin{verbatim}
SubjectLevelPermutationTestWithin_HMLET(
  data,
  samples = 2000,
  paired = T,
  threshold_t = NA
)
\end{verbatim}
\end{Usage}
\inputencoding{utf8}
\HeaderA{TrialLevelPermutationTestWithin\_HMLET}{Generate unique random permutations}{TrialLevelPermutationTestWithin.Rul.HMLET}
%
\begin{Description}\relax
Generate unique random permutations
\end{Description}
%
\begin{Usage}
\begin{verbatim}
TrialLevelPermutationTestWithin_HMLET(
  data,
  samples = 2000,
  paired = T,
  threshold_t = NA
)
\end{verbatim}
\end{Usage}
\inputencoding{utf8}
\HeaderA{UniquePermutations\_HMLET}{Generate unique random permutations}{UniquePermutations.Rul.HMLET}
%
\begin{Description}\relax
Generate unique random permutations
\end{Description}
%
\begin{Usage}
\begin{verbatim}
UniquePermutations_HMLET(listInput, n = 1)
\end{verbatim}
\end{Usage}
%
\begin{Arguments}
\begin{ldescription}
\item[\code{listInput}] Arbitrary list

\item[\code{n}] An integer number indicating the number of permutations
\end{ldescription}
\end{Arguments}
\printindex{}
\end{document}
